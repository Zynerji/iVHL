\documentclass[11pt,a4paper]{article}
\usepackage[utf8]{inputenc}
\usepackage[T1]{fontenc}
\usepackage{amsmath,amssymb}
\usepackage{graphicx}
\usepackage{hyperref}
\usepackage{listings}
\usepackage{xcolor}
\usepackage{geometry}
\usepackage{booktabs}
\usepackage{fancyhdr}

\geometry{margin=1in}

% Code listing style
\lstset{
    basicstyle=\ttfamily\small,
    backgroundcolor=\color{gray!10},
    frame=single,
    breaklines=true
}

% Header/Footer
\pagestyle{fancy}
\fancyhf{}
\rhead{iVHL Framework}
\lhead{\thepage}

% Title
\title{\textbf{iVHL: Integrated Vibrational Helix Lattice Framework} \\
       \large A Computational Research Platform for Quantum Gravity Phenomenology}
\author{iVHL Development Team}
\date{2025}

\begin{document}

\maketitle

\begin{center}
\texttt{Python 3.11} | \texttt{CUDA 12.5} | \texttt{Docker H100 Ready} | \texttt{MIT License}
\end{center}

\tableofcontents
\newpage

\section{Overview}

\textbf{iVHL} is a GPU-accelerated computational framework for exploring quantum gravity phenomenology through:

\begin{itemize}
\item \textbf{Holographic resonance simulations} with spherical boundary conditions
\item \textbf{Group Field Theory (GFT)} condensate dynamics and phase transitions
\item \textbf{Tensor network holography} (MERA, HaPPY codes, spin foams)
\item \textbf{LIGO-inspired gravitational wave} lattice analysis with fractal harmonics
\item \textbf{Reinforcement learning} discovery campaigns (TD3-SAC hybrid)
\item \textbf{Automated report generation} with LaTeX white papers
\end{itemize}

\textbf{Not a theory of everything} — a research tool for computational exploration of holographic duality, emergent geometry, and quantum-classical correspondence.

\section{Quick Start}

\subsection{Docker Deployment (H100 GPU - Recommended)}

\begin{lstlisting}[language=bash]
# Clone repository
git clone https://github.com/Zynerji/iVHL.git
cd iVHL

# Build H100-optimized container
docker build -t ivhl-h100:latest .

# Run with GPU support
docker run --gpus all -p 8501:8501 \
  -v $(pwd)/reports:/app/reports \
  -v $(pwd)/checkpoints:/app/checkpoints \
  ivhl-h100:latest

# Access Streamlit interface at http://localhost:8501
\end{lstlisting}

See \texttt{docs/DEPLOY\_H100.md} for complete deployment guide.

\subsection{Local Installation}

\begin{lstlisting}[language=bash]
# Install dependencies
pip install -r requirements.txt

# Run holographic resonance dashboard
streamlit run dashboards/resonance_dashboard.py

# Run GW lattice analysis
streamlit run dashboards/gw_dashboard.py
\end{lstlisting}

\section{Core Modules}

\subsection{Holographic Resonance (\texttt{ivhl/resonance/})}

Simulates wave interference on spherical boundary creating 3D resonant structures:

\textbf{Physics}:
\begin{equation}
\psi(\mathbf{r}, t) = \sum_i A_i \frac{\sin(k|\mathbf{r} - \mathbf{r}_i| + \phi_i(t))}{|\mathbf{r} - \mathbf{r}_i|}
\end{equation}

\begin{itemize}
\item Coherent wave sources on helical boundary lattice
\item 3D superposition with Calabi-Yau-like folded topologies
\item Phase singularities (vortices) with topological charges
\item Particle advection in resonant field
\end{itemize}

\textbf{Key Features}:
\begin{itemize}
\item GPU-accelerated field computation (PyTorch)
\item Multi-vortex dynamics with Fourier/RNN trajectory control
\item PyVista volumetric rendering
\item Streamlit interactive interface
\end{itemize}

\subsection{Group Field Theory (\texttt{ivhl/gft/})}

Implements GFT as UV-complete foundation for emergent spacetime:

\textbf{Colored Tensor Models}:
\begin{equation}
S[T] = \frac{1}{2}\text{Tr}(T^\dagger T) + \frac{\lambda}{d!}\text{Tr}(T^d)
\end{equation}

\begin{itemize}
\item Melonic diagram dominance in large-$N$ limit
\item Schwinger-Dyson equations for dressed propagator $G^*$
\item Critical coupling $\lambda_c$ and double-scaling limit
\end{itemize}

\textbf{GFT Condensate}:
\begin{equation}
V_{\text{eff}}(\sigma) = \frac{m^2}{2}|\sigma|^2 + \frac{\lambda}{d!}|\sigma|^d
\end{equation}

\begin{itemize}
\item Phase transition: non-geometric ($\sigma=0$) $\leftrightarrow$ geometric ($\sigma \neq 0$)
\item Gross-Pitaevskii dynamics
\item Emergent FLRW cosmology with bouncing solutions
\end{itemize}

\subsection{LIGO-Inspired GW Lattice (\texttt{ivhl/gw/})}

Explores gravitational wave phenomenology with vibrational lattice:

\textbf{Waveform Generation}:
\begin{itemize}
\item \textbf{Inspiral}: $f(t) \propto (t_c - t)^{-3/8}$ chirp
\item \textbf{Ringdown}: Quasinormal modes with Q-factor decay
\item \textbf{Stochastic}: $\Omega_{\text{GW}}(f) \propto f^{2/3}$ power-law spectrum
\item \textbf{Constant Lattice}: Embedded $\pi, e, \phi, \sqrt{2}, \sqrt{3}$ at harmonic frequencies
\end{itemize}

\textbf{Lattice Perturbation}:
\begin{align}
\Delta r &= r \cdot h(t) \quad\text{(radial strain)} \\
\frac{\Delta x}{x} &= h, \quad \frac{\Delta y}{y} = -h \quad\text{(tidal deformation)}
\end{align}

\textbf{Fractal Analysis}:
\begin{itemize}
\item Box-counting fractal dimension: $D_{\text{box}} = \lim_{\epsilon\to0} \frac{\log N(\epsilon)}{\log(1/\epsilon)}$
\item Harmonic series detection via FFT
\item Mathematical constant residue matching
\item Log-space power-law fitting
\end{itemize}

\subsection{Reinforcement Learning Discovery (\texttt{ivhl/rl/})}

RL-driven exploration of parameter space:

\textbf{TD3-SAC Hybrid}:
\begin{itemize}
\item Twin Delayed DDPG (TD3) for stability
\item Soft Actor-Critic (SAC) for exploration (entropy maximization)
\item Automatic temperature tuning ($\alpha$ auto-adjustment)
\item Prioritized experience replay
\end{itemize}

\textbf{GW-Specific Rewards}:
\begin{itemize}
\item Lattice stability (Procrustes similarity): weight 2.0
\item Fractal dimension (target $D \approx 2.6$): weight 1.5
\item Constant residues ($\pi, e, \phi$): weight 3.0
\item Attractor convergence (low variance): weight 2.5
\item Memory persistence (long $\tau$): weight 2.0
\end{itemize}

\textbf{Discovery Modes}:
\begin{enumerate}
\item \texttt{FIND\_CONSTANT\_LATTICE}
\item \texttt{FRACTAL\_HARMONIC\_STABILIZATION}
\item \texttt{ATTRACTOR\_CONVERGENCE}
\item \texttt{GW\_MEMORY\_FIELD}
\item \texttt{QUASINORMAL\_RINGING}
\item \texttt{LATTICE\_UNDER\_SCRAMBLING}
\end{enumerate}

\subsection{Automated Reporting (\texttt{ivhl/integration/})}

\textbf{Data Exfiltration \& Documentation}:
\begin{itemize}
\item JSON structured data
\item Markdown summary
\item LaTeX white paper with professional template
\item PDF compilation (pdflatex in Docker)
\item Automatic GitHub commit/push
\end{itemize}

\section{Performance Benchmarks}

\textbf{NVIDIA H100 80GB HBM3}:

\begin{table}[h]
\centering
\begin{tabular}{lcc}
\toprule
\textbf{Operation} & \textbf{Time} & \textbf{Speedup vs CPU} \\
\midrule
Holographic field $(8192^3)$ & 3-5 ms & $\sim$200$\times$ \\
MERA contraction (128 tensors) & 8-12 ms & $\sim$150$\times$ \\
GFT evolution $(64^3$ grid) & 1200-1800 ms & $\sim$100$\times$ \\
TD3-SAC update & 15-25 ms & $\sim$80$\times$ \\
GW waveform (4096 samples) & $\sim$10 ms & $\sim$50$\times$ \\
Fractal box-counting $(64^3)$ & $\sim$200 ms & $\sim$30$\times$ \\
\bottomrule
\end{tabular}
\caption{iVHL Framework Performance on H100}
\end{table}

\section{Project Structure}

\begin{lstlisting}
iVHL/
├── ivhl/                      # Core Python package
│   ├── resonance/            # Holographic resonance
│   ├── gft/                  # Group Field Theory
│   ├── tensor_networks/      # MERA, RT formula
│   ├── gw/                   # GW lattice analysis
│   ├── rl/                   # Reinforcement learning
│   ├── integration/          # API, reports
│   └── legacy/               # Deprecated modules
├── dashboards/               # Streamlit interfaces
├── simulations/              # Simulation scripts
├── scripts/                  # Utilities, benchmarks
├── tests/                    # Test suites
├── configs/                  # JSON configurations
├── docs/                     # Documentation
├── whitepapers/              # Generated PDFs
├── Dockerfile                # H100-optimized
└── README.md                 # This document
\end{lstlisting}

\section{Research Applications}

\subsection{Holographic Duality Exploration}
\begin{itemize}
\item Test AdS/CFT-like correspondences in discrete systems
\item Explore boundary-bulk information encoding
\item Investigate entanglement entropy scaling (Ryu-Takayanagi)
\end{itemize}

\subsection{Quantum Gravity Phenomenology}
\begin{itemize}
\item GFT phase transitions (non-geometric $\leftrightarrow$ geometric)
\item Spin foam amplitudes and CDT Hausdorff dimensions
\item Asymptotic safety RG flow fixed points
\end{itemize}

\subsection{Gravitational Wave Analysis}
\begin{itemize}
\item Fractal structure in stochastic GW background
\item Mathematical constant encoding in strain data
\item Memory field persistence (non-Markovian dynamics)
\end{itemize}

\section{Docker Self-Contained Deployment}

\textbf{Complete production-ready containerization}:

\begin{lstlisting}[language=bash]
# Build (includes all dependencies + LaTeX)
docker build -t ivhl-h100:latest .

# Run with automatic report generation
docker run --gpus all -p 8501:8501 \
  -v $(pwd)/reports:/app/reports \
  -v $(pwd)/checkpoints:/app/checkpoints \
  -v $(pwd)/logs:/app/logs \
  ivhl-h100:latest

# Reports auto-generated after simulations in:
# reports/report_YYYYMMDD_HHMMSS/
\end{lstlisting}

\section{References}

\subsection{Holographic Duality}
\begin{enumerate}
\item Maldacena, J. (1999). ``The Large N Limit of Superconformal Field Theories''. \textit{Int. J. Theor. Phys.} \textbf{38}: 1113-1133.
\item Ryu, S., Takayanagi, T. (2006). ``Holographic Derivation of Entanglement Entropy''. \textit{Phys. Rev. Lett.} \textbf{96}: 181602.
\item Pastawski, F. et al. (2015). ``Holographic quantum error-correcting codes''. \textit{JHEP} \textbf{06}: 149.
\end{enumerate}

\subsection{Group Field Theory}
\begin{enumerate}
\setcounter{enumi}{3}
\item Oriti, D. (2016). ``Group Field Theory and Loop Quantum Gravity''. \textit{Loop Quantum Gravity: The First 30 Years}.
\item Gielen, S., Oriti, D., Sindoni, L. (2013). ``Cosmology from Group Field Theory''. \textit{Phys. Rev. Lett.} \textbf{111}: 031301.
\item Gurau, R. (2011). ``Colored Group Field Theory''. \textit{Comm. Math. Phys.} \textbf{304}: 69-93.
\end{enumerate}

\subsection{Gravitational Waves}
\begin{enumerate}
\setcounter{enumi}{6}
\item Abbott, B.P. et al. (2016). ``Observation of Gravitational Waves''. \textit{Phys. Rev. Lett.} \textbf{116}: 061102.
\item LIGO Scientific Collaboration (2021). ``GWTC-3: Compact Binary Coalescences''. arXiv:2111.03606.
\end{enumerate}

\subsection{Tensor Networks}
\begin{enumerate}
\setcounter{enumi}{8}
\item Vidal, G. (2008). ``Class of Quantum Many-Body States That Can Be Efficiently Simulated''. \textit{Phys. Rev. Lett.} \textbf{101}: 110501.
\item Evenbly, G., Vidal, G. (2015). ``Tensor Network Renormalization''. \textit{Phys. Rev. Lett.} \textbf{115}: 180405.
\end{enumerate}

\section{Disclaimer}

iVHL is a \textbf{computational research tool} for exploring quantum gravity phenomenology. It is \textbf{not}:
\begin{itemize}
\item[$\times$] A theory of everything
\item[$\times$] A replacement for established physics
\item[$\times$] Claiming to have ``discovered new laws''
\end{itemize}

It \textbf{is}:
\begin{itemize}
\item[\checkmark] A framework for computational exploration
\item[\checkmark] A tool for testing holographic duality concepts
\item[\checkmark] A platform for RL-driven discovery
\item[\checkmark] An educational resource for quantum gravity formalisms
\end{itemize}

\textbf{Use responsibly} for research and education.

\section{License}

MIT License - See LICENSE file for details.

\section{Contact}

\begin{itemize}
\item \textbf{Repository}: \url{https://github.com/Zynerji/iVHL}
\item \textbf{Issues}: \url{https://github.com/Zynerji/iVHL/issues}
\item \textbf{Documentation}: See \texttt{docs/} directory
\end{itemize}

\section{Citation}

If using iVHL in research:

\begin{lstlisting}
@software{ivhl2025,
  title = {iVHL: Integrated Vibrational Helix Lattice
           Framework for Quantum Gravity Phenomenology},
  author = {iVHL Development Team},
  year = {2025},
  url = {https://github.com/Zynerji/iVHL},
  note = {Computational platform for holographic
          resonance, GFT, tensor networks, and
          LIGO-inspired GW analysis}
}
\end{lstlisting}

\vfill

\begin{center}
\textit{Computational exploration of holographic duality, \\
emergent geometry, and quantum-classical correspondence.}
\end{center}

\end{document}
